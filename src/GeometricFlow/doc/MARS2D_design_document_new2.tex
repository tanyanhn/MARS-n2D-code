\documentclass[a4paper,twoside]{ctexart}
\usepackage{geometry}
\geometry{margin=1cm,vmargin={0pt,1cm}}
\setlength{\topmargin}{-2cm}
\setlength{\paperheight}{23cm}
\setlength{\paperwidth}{18cm}
\setlength{\textheight}{19.6cm}
\setlength{\textwidth}{15cm}
\usepackage{makecell}
\usepackage{fancyhdr}
\usepackage{siunitx}
\usepackage{amssymb}
\usepackage{indentfirst}
\setlength{\parindent}{0.5em}

\pagenumbering{arabic}

% useful packages.
\usepackage{multirow}
\usepackage{caption}
\usepackage{mathrsfs}
\usepackage{amsfonts}
\usepackage{amsmath}
\usepackage{amsthm}
\usepackage{enumerate}
\usepackage{xcolor,graphicx,float}
\usepackage{epstopdf}
\usepackage{multicol}
\usepackage{fancyhdr}
\usepackage{layout}
\usepackage{listings}
\lstset{language=Matlab}
\lstset{breaklines}
\lstset{extendedchars=false}
\usepackage[colorlinks,linkcolor=blue]{hyperref}
\usepackage{xcolor}
\usepackage{cite}
\usepackage[numbers,sort&compress]{natbib} 
\setcitestyle{open={},close={}}
%\usepackage{natbibspacing}
%\renewcommand{\refname}{}
\usepackage{anyfontsize}
\usepackage{ulem}


% some common command
\newcommand{\dif}{\mathrm{d}}
\newcommand{\avg}[1]{\left\langle #1 \right\rangle}
\newcommand{\pdfrac}[2]{\frac{\partial #1}{\partial #2}}
\newcommand{\op}{\odot}
\newcommand{\Eabs}{E_{\mathrm{abs}}}
\newcommand{\Erel}{E_{\mathrm{rel}}}
\newcommand{\Ediv}{\mathrm{div}}%\div是除号
\newcommand{\lrq}[1]{\left( #1 \right)}
\newcommand{\avint}[1]{\frac{1}{\left|#1\right|}\int_{#1}}

\newcommand{\upcite}[1]{\textsuperscript{\textsuperscript{\cite{#1}}}} 

\makeatletter
\newcommand\sixteen{\@setfontsize\sixteen{17pt}{6}}
\renewcommand{\maketitle}{\bgroup\setlength{\parindent}{0pt}
\begin{flushleft}
\sixteen\bfseries \@title
\medskip
\end{flushleft}
\textit{\@author}
\egroup}
\makeatother

\CTEXsetup[format={\Large\bfseries}]{section}

\title{MARS2D \&几何流 程序设计文档}


\begin{document}
\maketitle
\noindent 下文中 \texttt{Vector} 表示 \texttt{std::vector},
\texttt{Point} 表示 \texttt{Vec<Real, Dim>}。根据测试结果,我们的两种
容器备选方案是\texttt{Vector}和IMV,所以不再需要\texttt{Container}模板,
统一用\texttt{Vector}来存储点列。\textcolor{red}{为了实现两种不
  同速度场的行为,在\texttt{TimeIntegrator}类中添加了一个模板参数
\texttt{template <int> class VectorRHS}来作为速度场的 policy 类,并通
过特例化来实现不同的行为。}

\section{class VectorFunction}
\begin{itemize}
    \item 函数$\mathbb{R}^{\texttt{Dim}}\times\mathbb{R}\rightarrow\mathbb{R}^{\texttt{Dim}}$的基类,可以作为速度场的基类使用。
    \item \textbf{模板:}\texttt{template<int Dim>}:\\\texttt{Dim} 表示空间维数。
    \item \textbf{成员函数:}
            \begin{enumerate}[(1)]
                \item \texttt{virtual const Point
                    operator()(}\textcolor{red}{\texttt{const Point \&pt}}\texttt{, Real t) const=0}\\
                \textbf{Private 成员函数}\\
                \textbf{输入:}\texttt{pt}表示\texttt{Dim}维空间中的一个点,\texttt{t}表示当前时间。\\
                \textbf{输出:}\texttt{pt}点处对应的速度场。\\
                \textbf{作用:}作为一个helper function使用。当速度场是给定的时,每个点对应的速度场函数都是相同的,所以没有必要去重载点列版本的速度场,只需要重载一个单点版本就可以满足。
                \item \texttt{virtual const Vector<Point> operator()(const Vector<Point> \&pts, Real t=0) const;}\\
                \textbf{输入:}\texttt{pts}表示一列\texttt{Point},\texttt{t}为当前时间。\\
                \textbf{输出:}点列中每个点对应的速度场。\\
                \textbf{作用:}提供一个默认的调用单点
                \texttt{operator()()}实现的版本,便于给定速度场的使用。
            \end{enumerate}
\end{itemize}
\textcolor{red}{
\section{class VectorOnHypersurface}}
\begin{itemize}
    \item 描述嵌入在\texttt{Dim}维空间中余维数是$1$的超曲面上的离散点列处的速度场,且在每个离散点处的速度场是依赖于其他离散点的。
    \item \textbf{模板:}\texttt{template<int Dim>}:\\
    \texttt{Dim}表示空间维数。
    \item \textcolor{red}{\sout{继承:\texttt{class DiscreteVecCoDimOne: public VectorFunction<Dim>}。}}
    \item \textbf{成员函数:}
            \begin{enumerate}[(1)]
                \item \texttt{virtual const Vector<Point> operator()(const Vector<Point> \&pts, Real t=0) const=0;}\\
                \textbf{输入:}\texttt{pts}表示一列\texttt{Point},
                \texttt{t}为当前时间。\\
                \textbf{输出:}点列中每个点对应的速度场。\\
                \textbf{作用:}求点列对应的速度场,需要在其继承类中实
                现。
                \item \texttt{virtual const Tensor<Real, 2> getJacobi(const Vector<Point> \&pts, Real t=0) const;}\\
                \textbf{输入:}\texttt{pts}表示一列\texttt{Point},\texttt{t}为当前时间。\\
                \textbf{输出:}用\texttt{Tensor<Real, 2>}表示的雅克比矩阵。\\
                \textbf{作用:}提供一个什么都不做的默认版本,因为可能
                有些情况下是不需要雅克比矩阵的。在其继承类中,若需要使用隐式方法进行界面追踪问题的求解,则
                  必须重载这一函数。暂时先调用LAPACKE来进行隐式方法
                    的求解。先得到一个正确的结果之后再寻找高效的求解分
                    块循环多对角线性方程组的算法。
            \end{enumerate}
\end{itemize}

\textcolor{red}{
\section{class VectorForCurvatureFlow}}
\begin{itemize}
    \item 曲率流的速度场类。
    \item \textbf{模板:}\texttt{template<int Dim, int Order>}:\\
    \texttt{Dim}表示空间维数,\texttt{Order}表示空间离散的阶数。
    \item \textbf{继承:}\texttt{class VectorForCurvatureFlow: public \textcolor{red}{VectorOnHypersurface<Dim>}}。
    \item \textbf{成员函数:}
        \begin{enumerate}[(1)]
            \item \texttt{const Vector<Point> operator()(const Vector<Point> \&pts, Real t=0) const;}\\
            \textbf{输入输出:}同基类一致。\\
            \textbf{作用:}根据曲率流对应阶数的空间离散方法,计算出示踪点列速度场的近似值。
            \item \texttt{const Tensor<Real, 2> getJacobi(const Vector<Point> \&pts, Real t=0) const;}\\
            \textbf{输入输出:}同基类一致。\\
            \textbf{作用:}根据曲率流对应阶数的空间离散方法,计算出示踪点列速度场对应的雅克比矩阵。
        \end{enumerate}
\end{itemize}

\textcolor{red}{
\section{class VectorForSurfaceDiffusionFlow}}
\begin{itemize}
    \item 表面扩散流的速度场类。
    \item \textbf{模板:}\texttt{template<int Dim, int Order>}:\\
    \texttt{Dim}表示空间维数,\texttt{Order}表示空间离散的阶数。
    \item \textbf{继承:}\texttt{class VectorForSurfaceDiffusionFlow: public \textcolor{red}{VectorOnHypersurface<Dim>}}。
    \item \textbf{成员函数:}
        \begin{enumerate}[(1)]
            \item \texttt{const Vector<Point> operator()(const Vector<Point> \&pts, Real t=0) const;}\\
            \textbf{输入输出:}同基类一致。\\
            \textbf{作用:}根据表面扩散流对应阶数的空间离散方法,计算出示踪点列速度场的近似值。
            \item \texttt{const Tensor<Real, 2> getJacobi(const Vector<Point> \&pts, Real t=0) const;}\\
            \textbf{输入输出:}同基类一致。\\
            \textbf{作用:}根据表面扩散流对应阶数的空间离散方法,计算出示踪点列速度场对应的雅克比矩阵。
        \end{enumerate}
\end{itemize}


\section{class TimeIntegrator}
\begin{itemize}
    \item 时间积分方法的基类。
    \item \textbf{模板:}\texttt{template<int
        Dim}\textcolor{red}{\texttt{, template <int> class
          VectorRHS}}\texttt{>}:\\\texttt{Dim} 表示空间维数
      \textcolor{red}{\texttt{, VectorRHS<Dim>}表示选择的速度场类型。}
    \item \textbf{成员函数:}
      \begin{enumerate}[(1)]
        \item 
           \texttt{virtual const Point timeStep(}\textcolor{red}{\texttt{const VectorRHS<Dim> \&v}}\texttt{,
            const Point \&pt, Real tn, Real k)=0:}\\
          \textbf{输入:}\texttt{v} 为速度场,\texttt{pt} 表示当前点,\texttt{tn} 为当前时间,\texttt{k} 为时间步长。\\
                \textbf{输出:}\texttt{pt} 在速度场 \texttt{v} 作用\texttt{k}时间后
                得到的新点。\\
                \textbf{作用:}函数使得 \texttt{pt} 在速度场
                \texttt{v} 的作用下运动 \texttt{k} 时间。需要在继承类
                中进行实现。\textcolor{red}{这里保留单点版本的接口是因为在
                \texttt{MARS2D}中,splitLongEdges 函数在速度场为
                \texttt{VectorFunction}的情形下的实现会用到时间积分方
                法的单点版本,把要加入的点先流映射再插入,可以进一步降
                低时间复杂度。}
                \item \texttt{virtual void
                    timeStep(}\textcolor{red}{\texttt{const VectorRHS<Dim> \&v}}\texttt{, Vector<Point> \&pts, Real tn, Real k)=0:}\\
                \textbf{输入:}\texttt{v} 为速度场,
                \texttt{pts} 表示一列\texttt{Point},\texttt{tn} 为当
                前时间,\texttt{k} 为时间步长。\\
                \textbf{输出:}\texttt{void},原址更改 \texttt{pts}。\\
                \textbf{作用:}函数使得 \texttt{pts} 中的一列点在速度
                场 \texttt{v} 的作用下运动 \texttt{k} 时间。需要在继承
                类中进行实现。
            \end{enumerate}
\end{itemize}

\section{ButcherTableau}
\subsection{namespace RK}
定义名字空间,不然\texttt{enum}中的\texttt{ERK}等会和\texttt{class ERK}等冲突。
\subsubsection{enum Type\_Major}
\begin{itemize}
    \item \texttt{enum Type\_Major\{ERK=0, DIRK, ARK, nRK\_Major\}}。
    \item \textbf{作用:}表示 RK 方法的一般类型。
\end{itemize}

\subsubsection{enum Type\_Minor}
\begin{itemize}
    \item \texttt{enum Type\_Minor\{ForwardEuler=0, ClassicRK4, SDIRK2, ESDIRK4, nRK\_Minor\}}。
    \item \textbf{作用:}表示 RK 方法的细分类型。
\end{itemize}

\subsection{struct ButcherTableau}
\begin{itemize}
    \item \textbf{模板:}\texttt{template<RK::Type\_Major Type1, RK::Type\_Minor Type2>}:\\\texttt{Type1} 表示 RK 的一般类型,如 ERK, DIRK, ARK 等;\texttt{Type2} 表示 RK 的细分类型,如 \texttt{Type1}是\texttt{ERK} 时,\texttt{Type2} 可以是 \texttt{ForwardEuler},\texttt{ClassicRK4}等;\texttt{Type2}是\texttt{DIRK}时,\texttt{Type2}可以是\texttt{SDIRK2},\texttt{ESDIRK4}等。对于给定的 \texttt{RK::Type\_Major} 中的一般类型,其数据结构是固定的。
    \item \textbf{特例化:}
            \begin{enumerate}[(1)]
                \item \texttt{template <>\\
                struct ButcherTableau<ERK, ForwardEuler>\\
                \{\\
                \hspace*{4pt} static constexpr int order = 1;\\
                \hspace*{4pt} static constexpr int nStages = 1;\\
                \hspace*{4pt} static constexpr Real a[nStages][nStages] = ...;\\
                \hspace*{4pt} static constexpr Real b[nStages] = ...;\\
                \hspace*{4pt} static constexpr Real c[nStages] = ...;\\
                \};}
            \item \texttt{template <>\\
                struct ButcherTableau<ERK, ClassicRK4>\\
                \{...\};}
            \item \texttt{template <>\\
                struct ButcherTableau<DIRK, SDIRK2>\\
                \{...\};}
            \item \texttt{template <>\\
                struct ButcherTableau<DIRK, ESDIRK4>\\
                \{...\};}
            \end{enumerate} 
\end{itemize}

\section{class ERK}
\begin{itemize}
    \item 继承自 \texttt{TimeIntegrator<Dim}\textcolor{red}{\texttt{, VectorRHS}}\texttt{>},用于实现所有 ERK 方法。
    \item \textbf{模板:}\texttt{template<int
        Dim}\textcolor{red}{\texttt{, template <int> class
          VectorRHS}}\texttt{, RK::Type\_Minor Type>}:\\\texttt{Dim}
      表示空间维数\textcolor{red}{\texttt{, VectorRHS<Dim>}表示选择的速度场类型},\texttt{Type} 是 ERK 方法的某个子方法。这里可以这么做是因为所有 ERK 方法的数据结构都是一致的,所以在 \texttt{class ERK} 中,只需要知道所用的是哪种子方法就可以建立对应的 Butcher 表,进而实现对应的 ERK 方法。
    \item \texttt{using ButcherTab = ButcherTableau<RK::ERK, Type>;\\
    const int order = ButcherTab::order;}
    \item \textbf{成员函数:}
      \begin{enumerate}[(1)]
        \item 
          \texttt{const Point timeStep(}\textcolor{red}{\texttt{const VectorRHS<Dim> \&v}}\texttt{,
            const Point \&pt, Real tn, Real k):}\\
          \textbf{输入输出:}同基类一致。\\
                \textbf{作用:}调用 \texttt{ButcherTab} 中的成员常量以
                及\texttt{v}的单点版本速度场实现对应的ERK方法,进一步
                实现基类中的纯虚函数。\textcolor{red}{只会在
                  \texttt{VectorRHS}为\texttt{VectorFunction}时使用,
                  若在\texttt{VectorRHS}为
                  \texttt{VectorOnHypersurface}时使用则会抛出异常。}
                \item \texttt{void timeStep(}\textcolor{red}{\texttt{const VectorRHS<Dim> \&v}}\texttt{, Vector<Point> \&pts, Real tn, Real k):}\\
                \textbf{输入输出:}同基类一致。\\
                \textbf{作用:}调用 \texttt{ButcherTab} 中的成员常量以及\texttt{v}的点列版本速度场实现对应的ERK方法,进一步实现基类中的纯虚函数。
            \end{enumerate}
\end{itemize}

\section{class DIRK}
\begin{itemize}
    \item 继承自 \texttt{TimeIntegrator<Dim}\textcolor{red}{\texttt{, VectorRHS}}\texttt{>},用于实现所有 DIRK 方法。
    \item \textbf{模板:}\texttt{template<int
        Dim}\textcolor{red}{\texttt{, template <int> class
          VectorRHS}}\texttt{, RK::Type\_Minor Type>}:\\\texttt{Dim}
      表示空间维数\textcolor{red}{\texttt{, VectorRHS<Dim>}表示选择的速度场类型},\texttt{Type} 是 DIRK 方法的某个子方法。这里可以这么做是因为所有 DIRK 方法的数据结构都是一致的,系数矩阵\texttt{a}都是下三角的,所以在 \texttt{class DIRK} 中,只需要知道所用的是哪种子方法就可以建立对应的 Butcher 表,进而实现对应的 DIRK 方法。
    \item \texttt{using ButcherTab = ButcherTableau<RK::DIRK, Type>;\\
    const int order = ButcherTab::order;}
    \item \textbf{成员函数:}
      \begin{enumerate}[(1)]
        \item \texttt{const Point timeStep(}\textcolor{red}{\texttt{const VectorRHS<Dim> \&v}}\texttt{,
            const Point \&pt, Real tn, Real k):}\\
          \textbf{输入输出:}同基类一致。\\
                \textbf{作用:}调用 \texttt{ButcherTab} 中的成员常量以
                及\texttt{v}的单点版本速度场实现对应的DIRK方法,进一步
                实现基类中的纯虚函数。\textcolor{red}{只会在
                  \texttt{VectorRHS}为\texttt{VectorFunction}时使用,
                  若在\texttt{VectorRHS}为
                  \texttt{VectorOnHypersurface}时使用则会抛出异常。}
                \item \texttt{void timeStep(}\textcolor{red}{\texttt{const VectorRHS<Dim> \&v}}\texttt{, Vector<Point> \&pts, Real tn, Real k):}\\
                \textbf{输入输出:}同基类一致。\\
                \textbf{作用:}调用 \texttt{ButcherTab} 中的成员常量以
                及\texttt{v}的点列版本速度场和\texttt{getJaocbi}函数来实现对应的DIRK方法,进一步实现基类中的纯虚函数。过程中需要使用牛顿迭代求解非线性方程组。
            \end{enumerate}
\end{itemize}


\section{class MARS}
\begin{itemize}
    \item 殷集界面追踪方法的基类。
    \item \textbf{模板:}\texttt{template<int Dim, int Order}\textcolor{red}{\texttt{,
        template<int> class VelocityField}}\texttt{>}:\\
    其中 \texttt{Dim} 表示维数,\texttt{Order} 表示殷集边界表示的阶数\textcolor{red}{,
    \texttt{VelocityField<Dim>} 表示选择的速度场类型}。
    \item \textbf{成员变量:}
        \begin{enumerate}[(1)]
            \item \texttt{TimeIntegrator<Dim}\textcolor{red}{\texttt{,
        VelocityField<Dim> }}\texttt{> *TI:}\textbf{\texttt{Protected}  成员变量},时间积分方法基类指针。
        \end{enumerate}
    \item \textbf{成员函数:}
        \begin{enumerate}[(1)]
            \item \texttt{MARS(TimeIntegrator<Dim}\textcolor{red}{\texttt{,
        VelocityField<Dim> }}\texttt{> *\_TI):TI(\_TI)\{\}:}\\构造函数。
            \item \texttt{virtual void
                timeStep(}\textcolor{red}{\texttt{const VelocityField<Dim> \&v}}\texttt{, YinSet<Dim, Order> \&ys, Real tn, Real k) = 0:}\\
            \textbf{输入:}\texttt{v} 为速度场,\texttt{ys} 为殷集,\texttt{tn} 和 \texttt{k} 的定义同时间积分方法中一致。\\
            \textbf{输出:}\texttt{void},原址更改 \texttt{ys}。\\
            \textbf{作用:}纯虚函数,作为二维和三维 MARS 方法 \texttt{timeStep} 的公共接口,在继承类中进行实现。函数将 \texttt{tn} 时刻的殷集映射到 \texttt{k} 时间后的殷集并赋值给 \texttt{ys}。
            \item \texttt{void
                trackInterface(}\textcolor{red}{\texttt{const VelocityField<Dim> \&v}}\texttt{, YinSet<Dim,Order> \&ys, Real startTime, Real k, Real endTime):}\\
            \textbf{输入:}\texttt{v} 为速度场,\texttt{ys} 为殷集,\texttt{startTime} 和 \texttt{endTime} 表示 MARS 方法作用的起止时间,\texttt{k} 为时间步长。\\
            \textbf{输出:}\texttt{void},原址更改 \texttt{ys}。\\
            \textbf{作用:}在速度场 \texttt{v} 的作用下,将 \texttt{startTime} 时刻的殷集 \texttt{ys} 通过时间步长为 \texttt{k} 的 MARS 方法映射到 \texttt{endTime} 时刻的殷集并赋值给 \texttt{ys}。调用 \texttt{timeStep} 在此基类中进行实现。
        \end{enumerate}
\end{itemize}


\section{class MARS2D}
\begin{itemize}
    \item 二维殷集的界面追踪方法。
    \item \textbf{模板:}\texttt{template<int Order}\textcolor{red}{\texttt{,
        template <int> class VelocityField}}\texttt{>}:\\
    其中 \texttt{Order} 表示二维殷集边界所用样条曲线的阶数\textcolor{red}{,
    \texttt{VelocityField<2>} 表示选择的速度场类型}。
    \item \textbf{继承:}\texttt{class MARS2D: public MARS<2, Order}\textcolor{red}{\texttt{,
        VelocityField}}\texttt{>}。\\
    \texttt{using Base = MARS<2, Order}\textcolor{red}{\texttt{,
        VelocityField}}\texttt{>}。
    \item \textbf{成员变量:}
            \begin{enumerate}[(1)]
                \item \texttt{Interval<1> chdLenRange:}殷集边界上相邻节点间弦长取值范围。
            \end{enumerate}
    \item \textbf{成员函数:}
            \begin{enumerate}[(1)]
                \item \texttt{MARS2D(TimeIntegrator<2}\textcolor{red}{\texttt{,
        VelocityField<2> }}\texttt{> *\_TI, Real hL, Real rtiny=0.1):\\Base(\_TI)\{...\}:}\\
                构造函数,使 \texttt{chdLenRange} 为\texttt{[rtiny*hL,hL]},\texttt{rtiny} 默认值为 0.1。
                \item \texttt{void
                    discreteFlowMap(}\textcolor{red}{\texttt{const VelocityField<2> \&v}}\texttt{, Vector<Point> \&pts, Real tn, Real k):}\\
                  \textbf{Private 成员函数}\\
                \textbf{输入:}\texttt{v} 为速度场,\texttt{pts} 为示踪点列,\texttt{tn} 和 \texttt{k} 的定义同时间积分方法中的一致。\\
                \textbf{输出:}\texttt{void},原址更改 \texttt{pts}。\\
                \textbf{作用:}函数调用 \texttt{Base::TI->timeStep} 将 \texttt{pts} 映射到 \texttt{k} 时间后的示踪点列。
                \item \texttt{Vector<unsigned int> splitLongEdges(}\textcolor{red}{\texttt{const VelocityField<2> \&v}}\texttt{, Vector<Point> \&pts, const Curve<2, Order> \&crvtn, Real tn, Real k):}\\
                  \textbf{Private 成员函数}\\
                \textbf{输入:}\texttt{v} 为速度场,\texttt{pts} 为下一时刻的示踪点列,\texttt{crvtn} 为当前时刻对应的样条曲线,\texttt{tn} 和 \texttt{k} 的定义同时间积分方法中的一致。\\
                \textbf{输出:}\texttt{pts} 中新加入点的序号。\\
                \textbf{作用:}函数首先从\texttt{crvtn}中提取出当前时刻对应的示踪点列\texttt{oldpts},之后在\texttt{pts}中寻找相邻点间距离过长的弦,并在\texttt{crvtn}中对应的曲线段上加点得到加点后\texttt{oldpts},利用 \texttt{Base::TI->timeStep} 将新加入的点映射到下一时刻得到新的\texttt{pts}。
                \item \texttt{Vector<unsigned int> removeSmallEdges(Vector<Point> \&pts):}\\
                \textbf{Private 成员函数}\\
                \textbf{输入:}\texttt{pts} 为下一时刻的示踪点列。\\
                \textbf{输出:}删除的节点在原 pts 中的序号。\\
                \textbf{作用:}函数将 \texttt{pts} 中相邻点间弦长过小的点在原址进行删除,满足不删除首尾两点、不连续删点的条件。
                \item \texttt{void timeStep(}\textcolor{red}{\texttt{const VelocityField<2> \&v}}\texttt{, YinSet<2,Order> \&ys, Real tn, Real k):}\\
                \textbf{输入:}同基类一致。\\
                \textbf{输出:}同基类一致。\\
                \textbf{作用:}实现基类中的纯虚函数,函数依次调用
                discreteFlowMap, splitLongEdges, removeSmallEdges 和
                fitCurve 实现二维 MARS 方法中的一个时间步,即预处理、
                流映射、后处理的复合,将当前时刻的殷集映射到
                \texttt{k} 时间后的殷集并赋值给 \texttt{ys}。
            \end{enumerate}
          \end{itemize}




%\begin{figure}[!htbp]
%  \centering
%  \includegraphics[angle=90,wikh = 0.6\textwikh]{2DMARS}
%  \caption{2DMARS框架}
%\end{figure}

\end{document}
