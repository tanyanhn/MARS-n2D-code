\documentclass[a4paper,twoside]{ctexart}
\usepackage{geometry}
\geometry{margin=1cm,vmargin={0pt,1cm}}
\setlength{\topmargin}{-2cm}
\setlength{\paperheight}{23cm}
\setlength{\paperwidth}{18cm}
\setlength{\textheight}{19.6cm}
\setlength{\textwidth}{15cm}
\usepackage{makecell}
\usepackage{fancyhdr}
\usepackage{siunitx}
\usepackage{amssymb}
\usepackage{indentfirst}
\setlength{\parindent}{0.5em}

\pagenumbering{arabic}

% useful packages.
\usepackage{multirow}
\usepackage{caption}
\usepackage{mathrsfs}
\usepackage{amsfonts}
\usepackage{amsmath}
\usepackage{amsthm}
\usepackage{enumerate}
\usepackage{xcolor,graphicx,float}
\usepackage{epstopdf}
\usepackage{multicol}
\usepackage{fancyhdr}
\usepackage{layout}
\usepackage{listings}
\lstset{language=Matlab}
\lstset{breaklines}
\lstset{extendedchars=false}
\usepackage[colorlinks,linkcolor=blue]{hyperref}
\usepackage{xcolor}
\usepackage{cite}
\usepackage[numbers,sort&compress]{natbib} 
\setcitestyle{open={},close={}}
%\usepackage{natbibspacing}
%\renewcommand{\refname}{}
\usepackage{anyfontsize}


% some common command
\newcommand{\dif}{\mathrm{d}}
\newcommand{\avg}[1]{\left\langle #1 \right\rangle}
\newcommand{\pdfrac}[2]{\frac{\partial #1}{\partial #2}}
\newcommand{\op}{\odot}
\newcommand{\Eabs}{E_{\mathrm{abs}}}
\newcommand{\Erel}{E_{\mathrm{rel}}}
\newcommand{\Ediv}{\mathrm{div}}%\div是除号
\newcommand{\lrq}[1]{\left( #1 \right)}
\newcommand{\avint}[1]{\frac{1}{\left|#1\right|}\int_{#1}}

\newcommand{\upcite}[1]{\textsuperscript{\textsuperscript{\cite{#1}}}} 

\makeatletter
\newcommand\sixteen{\@setfontsize\sixteen{17pt}{6}}
\renewcommand{\maketitle}{\bgroup\setlength{\parindent}{0pt}
\begin{flushleft}
\sixteen\bfseries \@title
\medskip
\end{flushleft}
\textit{\@author}
\egroup}
\makeatother

\CTEXsetup[format={\Large\bfseries}]{section}

\title{MARS2D 程序设计文档}


\begin{document}
\maketitle
\noindent 下文中 \texttt{Vector} 表示 \texttt{std::vector},\texttt{List} 表示 \texttt{std::list},\texttt{Point} 表示 \texttt{Vec<Real, Dim>}。{\color{blue}蓝色的文字表示是为了能够同时实现 \texttt{Vector} 和 \texttt{List} 而改变的。}

\section{\texttt{Vector} 和 \texttt{List} 的区别}
\subsection{\texttt{Vector}}
在内存中分配一块连续的内存空间进行存储,相对于\texttt{List}更节省空间。支持随机访问,能通过\texttt{operator[]}进行对数据成员的访问,随机访问的时间复杂度为$\mathrm{O}(1)$,并且其\texttt{iterator}支持\texttt{operator+}操作。但在进行插入和删除操作时会造成内存块的拷贝,效率较低,时间复杂度为$\mathrm{O}(n)$。

\subsection{\texttt{List}}
每个数据节点包括一个信息块、一个前驱指针和一个后驱指针,便于进行序列内部的插入和删除操作,时间复杂度为$\mathrm{O}(1)$。但不支持随机访问,访问指定数据的时间复杂度为$\mathrm{O}(n)$,其\texttt{iterator}也只支持\texttt{operator++}。在创建大体量的数据时, \texttt{List} 比 \texttt{Vector} 要慢很多,因为其内存不是连续的。

\subsection{将\texttt{Vector}替换为\texttt{List}对程序性能的影响}
首先由于加减点时会调用容器的 \texttt{emplace}(插入成员函数)和 \texttt{erase}(删除成员函数),所以使用\texttt{List}在这方面可以降低时间复杂度。在随机访问方面,程序中仅在计算相邻点间弦长时调用了点列中的点坐标信息,而能否随机访问并不影响其时间复杂度。而对于\texttt{iterator}是否重载了\texttt{operator+},程序中在进行加点时确实调用了\texttt{Vector}迭代器的\texttt{operaotr+},但可以通过循环调用\texttt{operator++}解决,时间复杂度应和加点个数同阶。在创建临时变量方面,\texttt{List} 会更慢一些,具体影响程度还需验证。

在实现上,程序中统一使用 \texttt{iterator} 来进行点列的遍历;\texttt{Vector} 和 \texttt{List} 的 \texttt{emplace}、\texttt{erase} 接口是一样的,不需要改变;在进行 \texttt{iterator} 移位时,统一用循环调用 \texttt{operator++} 来实现。这样一套程序就可以同时实现 \texttt{Vector} 和 \texttt{List} 两个版本。

\section{class VectorFunction}
\begin{itemize}
    \item 函数$\mathbb{R}^{\texttt{Dim}}\times\mathbb{R}\rightarrow\mathbb{R}^{\texttt{Dim}}$的基类,可以作为速度场的基类使用。
    \item \textbf{模板:}\texttt{template<int Dim>}:\\\texttt{Dim} 表示空间维数。
    \item \textbf{成员函数:}
            \begin{enumerate}[(1)]
                \item \texttt{virtual const Point operator()(const Point \&pt, Real t) const = 0:}\\
                \textbf{输入:}\texttt{pt} 为当前点坐标,\texttt{t} 为当前时间。\\
                \textbf{输出:}\texttt{pt} 点处的速度场。\\
            \end{enumerate}
\end{itemize}

\section{class TimeIntegrator}
\begin{itemize}
    \item 时间积分方法的基类。
    \item \textbf{模板:}\texttt{template<int Dim>}:\\\texttt{Dim} 表示空间维数。
    \item \textbf{成员函数:}
            \begin{enumerate}[(1)]
                \item \texttt{virtual const Point timeStep(const VectorFunction<Dim> \&v, const Point \&pt, Real tn, Real k) = 0:}\\
                \textbf{输入:}\texttt{v} 为速度场,\texttt{pt} 为当前点坐标,\texttt{tn} 为当前时间,\texttt{k} 为时间步长。\\
                \textbf{输出:}新的点坐标。\\
                \textbf{作用:}使得 \texttt{pt} 在速度场 \texttt{v} 的作用下运动 \texttt{k} 时间。
                \item \texttt{template<template<typename...>class Container>\\void timeStep(const VectorFunction<Dim> \&v, Container<Point> \&pts, Real tn, Real k):}\\
                \textbf{输入:}\texttt{v} 为速度场,\texttt{pts} 表示一列 Point ,\text{tn} 为当前时间,\texttt{k} 为时间步长。\\
                \textbf{输出:}\texttt{void},原址更改 pts。\\
                \textbf{作用:}函数使得 pts 中的一列点在速度场 \texttt{v} 的作用下运动 \texttt{k} 时间,调用单点版本的 \texttt{timeStep} 成员函数进行实现。
            \end{enumerate}
\end{itemize}

\section{ButcherTableau}
\subsection{enum RK\_Category1}
\begin{itemize}
    \item \texttt{enum RK\_Category1\{ERK=1, DIRK, ARK, nRK\_Family\}}。
    \item \textbf{作用:}表示 RK 方法的一般类型。
\end{itemize}

\subsection{enum RK\_Category2}
\begin{itemize}
    \item \texttt{enum RK\_Category2\{ForwardEuler=1, ClassicRK4, nRK\_Type\}}。
    \item \textbf{作用:}表示 RK 方法的细分类型。
\end{itemize}

\subsection{struct ButcherTableau}
\begin{itemize}
    \item \textbf{模板:}\texttt{template<RK\_Category1 Type1, RK\_Category2 Type2>}:\\\texttt{Type1} 表示 RK 的一般类型,如 ERK, DIRK, ARK 等;\texttt{Type2} 表示 RK 的细分类型,如 \texttt{Type1}=ERK 时,\texttt{Type2} 可以是 ForwardEuler, ClassicRK4 等。对于给定的 \texttt{RK\_Category1} 中的一般类型,其数据结构是固定的。
    \item \textbf{特例化:}
            \begin{enumerate}[(1)]
                \item \texttt{template <>\\
                struct ButcherTableau<ERK, ForwardEuler>\\
                \{\\
                \hspace*{4pt} static constexpr int nStages = 1;\\
                \hspace*{4pt} static constexpr Real a[nStages][nStages] = ...;\\
                \hspace*{4pt} static constexpr Real b[nStages] = ...;\\
                \hspace*{4pt} static constexpr Real c[nStages] = ...;\\
                \};}
            \item \texttt{template <>\\
                struct ButcherTableau<ERK, ClassicRK4>\\
                \{...\};}
            \end{enumerate} 
\end{itemize}

\section{class ExplicitRungeKutta}
\begin{itemize}
    \item 继承自 \texttt{TimeIntegrator<Dim>},用于实现所有 ERK 方法。
    \item \textbf{模板:}\texttt{template<int Dim, RK\_Category2 Type>}:\\\texttt{Dim} 表示空间维数,\texttt{Type} 是 ERK 方法的某个子方法。这里可以这么做是因为所有 ERK 方法的数据结构都是一致的,所以在 \texttt{class ExplicitRungeKutta} 中,只需要知道所用的是哪种子方法就可以建立对应的 Butcher 表,进而实现对应的 ERK 方法。
    \item \texttt{using ButcherTab = ButcherTableau<ERK, Type>;}
    \item \textbf{成员函数:}
            \begin{enumerate}[(1)]
                \item \texttt{const Point timeStep(const VectorFunction<Dim> \&v, const Point \&pt, Real tn, Real k):}\\
                \textbf{输入:}同时间积分方法中的纯虚函数 \texttt{timeStep} 一致。\\
                \textbf{输出:}同时间积分方法中的纯虚函数 \texttt{timeStep} 一致。\\
                \textbf{作用:}调用 \texttt{ButcherTab} 中的成员常量实现对应的 ERK 方法,进一步实现 \texttt{TimeInteg-\\rator<Dim>} 中的纯虚函数 \texttt{timeStep}。
            \end{enumerate}
\end{itemize}

\section{class MARS}
\begin{itemize}
    \item 殷集界面追踪方法的基类。
    \item \textbf{模板:}\texttt{template<int Dim, int Order>}:\\
    其中 \texttt{Dim} 表示维数,\texttt{Order} 表示殷集边界表示的阶数。
    \item \textbf{成员变量:}
        \begin{enumerate}[(1)]
            \item \texttt{TimeIntegrator<Dim> *TI:}\textbf{\texttt{Protected}  成员变量},时间积分方法基类指针。
        \end{enumerate}
    \item \textbf{成员函数:}
        \begin{enumerate}[(1)]
            \item \texttt{MARS(TimeIntegrator<Dim> *\_TI):TI(\_TI)\{\}:}\\构造函数。
            \item \texttt{virtual void timeStep(const VectorFunction<Dim> \&v, YinSet<Dim, Order> \&ys, Real tn, Real k) = 0:}\\
            \textbf{输入:}\texttt{v} 为速度场,\texttt{ys} 为殷集,\texttt{tn} 和 \texttt{k} 的定义同时间积分方法中一致\\
            \textbf{输出:}\texttt{void},原址更改 \texttt{ys}。\\
            \textbf{作用:}纯虚函数,作为二维和三维 MARS 方法 \texttt{timeStep} 的公共接口,在继承类中进行实现。函数将 \texttt{tn} 时刻的殷集映射到 \texttt{k} 时间后的殷集并赋值给 \texttt{ys}。
            \item \texttt{void trackInterface(const VectorFunction<Dim> \&v, YinSet<Dim,Order> \&ys, Real startTime, Real k, Real endTime):}\\
            \textbf{输入:}\texttt{v} 为速度场,\texttt{ys} 为殷集,\texttt{startTime} 和 \texttt{endTime} 表示 MARS 方法作用的起止时间,\texttt{k} 为时间步长。\\
            \textbf{输出:}\texttt{void},原址更改 \texttt{ys}。\\
            \textbf{作用:}在速度场 \texttt{v} 的作用下,将 \texttt{startTime} 时刻的殷集 \texttt{ys} 通过时间步长为 \texttt{k} 的 MARS 方法映射到 \texttt{endTime} 时刻的殷集并赋值给 \texttt{ys}。调用 \texttt{timeStep} 在此基类中进行实现。
        \end{enumerate}
\end{itemize}


\section{class MARS2D}
\begin{itemize}
    \item 二维殷集的界面追踪方法。
    \item \textbf{模板:}\texttt{template<int Order, {\color{blue}template<typename...> class Container}>}:\\
    其中 \texttt{Order} 表示二维殷集边界所用样条曲线的阶数,{\color{blue}\texttt{Container} 表示某种 STL 容器,用于存储示踪点列,目前考虑 \texttt{Vector} 和 \texttt{List} 两种。}
    \item \textbf{继承:}\texttt{class MARS2D: public MARS<2, Order>}。\\
    \texttt{using Base = MARS<2, Order>。}
    \item \textbf{成员变量:}
            \begin{enumerate}[(1)]
                \item \texttt{Interval<1> chdLenRange:}殷集边界上相邻节点间弦长取值范围。
            \end{enumerate}
    \item \textbf{成员函数:}
            \begin{enumerate}[(1)]
                \item \texttt{MARS2D(TimeIntegrator<2> *\_TI, Real hL, Real rtiny=0.1):Base(\_TI)\{...\}:}\\
                构造函数,使 \texttt{chdLenRange} 为\texttt{[rtiny*hL,hL]},\texttt{rtiny} 默认值为 0.1。
                \item \texttt{void discreteFlowMap(const VectorFunction<2> \&v, {\color{blue}Container<Point> \&pts}, \\Real tn, Real k):}\\
                \textbf{Private 成员函数}\\
                \textbf{输入:}\texttt{v} 为速度场,\texttt{pts} 为示踪点列,\texttt{tn} 和 \texttt{k} 的定义同时间积分方法中的一致。\\
                \textbf{输出:}\texttt{void},原址更改 \texttt{pts}。\\
                \textbf{作用:}函数调用 \texttt{Base::TI->timeStep} 的 \texttt{Container} 版本将 \texttt{pts} 映射到 \texttt{k} 时间后的示踪点列。
                \item \texttt{Vector<unsigned int> splitLongEdges(const VectorFunction<2> \&v, \\{\color{blue}Container<Point> \&pts}, const Curve<2, Order> \&crvtn, Real tn, Real k):}\\
                \textbf{Private 成员函数}\\
                \textbf{输入:}\texttt{v} 为速度场,\texttt{pts} 为下一时刻的示踪点列,\texttt{crvtn} 为当前时刻对应的样条曲线,\texttt{tn} 和 \texttt{k} 的定义同时间积分方法中的一致。\\
                \textbf{输出:}\texttt{pts} 中新加入点的序号。\\
                \textbf{作用:}函数在 \texttt{pts} 中寻找相邻点间距离过长的弦,并在 \texttt{crvtn} 中对应的曲线段上加点,利用 \texttt{Base::TI->timeStep} 将新加入的点映射到下一时刻并添加到 \texttt{pts} 中(注意这里调用的是 \texttt{timeStep} 的单点映射版本,避免对本身就存在于 \texttt{pts} 中的点进行重复操作)。
                \item \texttt{Vector<unsigned int> removeSmallEdges({\color{blue}Container<Point> \&pts}):}\\
                \textbf{Private 成员函数}\\
                \textbf{输入:}\texttt{pts} 为下一时刻的示踪点列。\\%,{\color{red}\texttt{debug} 表示是否处于调试模式。加入 \texttt{debug} 是因为目前的 \texttt{removeSmallEdges} 中建立了一个临时的 \texttt{List<int> ids} ,其元素对应节点序号,在删点的同时删除 \texttt{ids} 中的元素以标记剩余点的序号,进一步得到删除点的序号。而删除节点的序号只会在调试阶段使用,所以这里希望在非调试阶段只输出删除点的个数,避免临时 \texttt{List} 变量的创建。}\\
                \textbf{输出:}删除的节点在原 pts 中的序号。\\%{\color{red}或仅包含一个元素的 \texttt{Vector},该元素为删除点的个数}。\\
                \textbf{作用:}函数将 \texttt{pts} 中相邻点间弦长过小的点在原址进行删除,满足不删除首尾两点、不连续删点的条件。
                \item \texttt{void timeStep(const VectorFunction<2> \&v, YinSet<2,Order> \&ys, Real tn, Real k):}\\
                \textbf{输入:}同基类一致。\\
                \textbf{输出:}同基类一致。\\
                \textbf{作用:}实现基类中的纯虚函数,函数依次调用 discreteFlowMap, splitLongEdges, removeSmallEdges 和 fitCurve 实现二维 MARS 方法中的一个时间步,即预处理、流映射、后处理的复合,将当前时刻的殷集映射到 \texttt{k} 时间后的殷集并赋值给 \texttt{ys}。
            \end{enumerate}
          \end{itemize}




%\begin{figure}[!htbp]
%  \centering
%  \includegraphics[angle=90,wikh = 0.6\textwikh]{2DMARS}
%  \caption{2DMARS框架}
%\end{figure}

\end{document}